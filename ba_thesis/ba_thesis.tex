%use the KomaScript styling on A4 paper
\documentclass[a4paper]{scrreprt}
%package for citing in Harvard style
\usepackage{natbib}
\bibliographystyle{agsm}
\usepackage[T1]{fontenc}
\usepackage[utf8]{inputenc}
\usepackage{graphicx}
\graphicspath{{"./figs"}}

\begin{document}

\begin{titlepage}
\centering
{\large\textsc{Bachelor thesis}\par}
\vspace{8\baselineskip}
{\Huge The right-wing populism of the AfD party and its regional differences\par}
\vspace{2\baselineskip}
{\Large A quantitative analysis on populist antagonisms within the programmatic discourse\par}
\vspace{5\baselineskip}
{\large\textsc{submitted by\\[.5em]Dipl.-Math. Paul Keydel}\par}
\vspace{8\baselineskip}
{Freie Universität Berlin, April 2024\par}
\vfill
\raggedright
{\em 1st supervisor: Dr. Julia Reuschenbach\\}
{\em 2nd supervisor: Prof. Dr. Bruno Castanho Silva}
\end{titlepage}

\tableofcontents

%%%%%
% 1 %
%%%%%
\chapter{Introduction}
The aim of this thesis is to bla bla
%%%%%
% 2 %
%%%%%
\chapter{Populism as ideology of the democracy}
\section{What is an ideology?}
In order to understand the connection between populism and ideology, this section will outline a theoretical framework behind ideologies.
\section{Populism: the bridge between \guilsinglright the people\guilsinglleft\ and \guilsinglright the politics\guilsinglleft}
Both left-wing (Mouffe) and right-wing populism construct a version of "the poeple".\\
right-wing populism: aims for a simplification of reality\\
left-wing populism (Mouffe): shall clarify structures of oppression and exploitation
\section{The role of antagonisms}
Antagonisms either help to reduce the whole world or to re-politicize society.\\
right-wing populism: Fight against immigrants, political and economic elites\\
left-wing populism (Mouffe): Fight against poverty, the capital, privileged classes
%%%%%
% 3 %
%%%%%
\chapter{The AfD party and its right-wing populism}
\section{Usage of right-wing antagonisms at the level of the federal states}
The AfD party changed from Euroscepticism to (radical) right. Populist strategies against the government are common, also against immigration (Remigration).\\
Example of antagonistic speech: Telegram messages of the highest reaction rates!\\
Rating of the Verfassungsschutz concerning right-wing extremism is a potential East-West difference. The more extremistic the more concentration on ethnic aspects.\\
With this in mind, one might ask the question of whether the AfD party creates region-dependent populist narratives or a unique overall narrative related to right-wing antagonisms.
\begin{figure}
    \centering
    \includegraphics[width=0.9\textwidth]{telegram_wc_most_feedbacked.pdf}
    \caption{a nice plot}
\end{figure}
\section{Towards an analysis of the programmatic discourse}
Limit the analysis to the four subdicts from RPC. Why are these antagonisms relevant to the question in section 3.1.?\\
How can we achieve regional information on programmatic development? Manifestos are in this sense very helpful because they are created within a democratic process and therefore offer a deeper view inside the party.\\
Is there other current research that uses AfD manifestos? UK comparison, etc...\\
Which manifestos had specifically been chosen for this analysis, and why?
%%%%%
% 4 %
%%%%%
\chapter{An analysis of antagonisms within AfD manifestos}
\section{A dictionary-based approach with \em RPC-Lex}
What are dictionaries and how can they be used to measure populism?\\
What is the RPC-Lex and what is it used for? How was it constructed/collected?\\
Cite quanteda!
\begin{figure}
    \centering
    \includegraphics[width=0.9\textwidth]{manifestos_wordcloud_disjoint_subdicts.pdf}
    \caption{a nice plot}
\end{figure}
\section{Results for the last two state elections}
\begin{figure}
    \centering
    \includegraphics[width=\textwidth]{manifestos_dict_states_2013.pdf}
    \caption{a nice plot}
\end{figure}
\begin{figure}
    \centering
    \includegraphics[width=\textwidth]{manifestos_dict_states_2018.pdf}
    \caption{a nice plot}
\end{figure}
\section{Temporal changes between East and West Germany}
\begin{figure}
    \centering
    \includegraphics[width=\textwidth]{manifestos_dict_states_diffs.pdf}
    \caption{a nice plot}
\end{figure}
\begin{figure}
    \centering
    \includegraphics[width=\textwidth]{manifestos_diffs_pca.pdf}
    \caption{a nice plot}
\end{figure}
\begin{figure}
    \centering
    \includegraphics[width=\textwidth]{manifestos_temporal_regression_west.pdf}
    \caption{a nice plot}
\end{figure}
\begin{figure}
    \centering
    \includegraphics[width=\textwidth]{manifestos_temporal_regression_east.pdf}
    \caption{a nice plot}
\end{figure}
%%%%%
% 5 %
%%%%%
\chapter{Evaluation of the findings and validation}
\begin{figure}
    \centering
    \includegraphics[width=0.9\textwidth]{manifestos_wordcloud_diffs_west.pdf}
    \caption{a nice plot}
\end{figure}
\begin{figure}
    \centering
    \includegraphics[width=0.9\textwidth]{manifestos_wordcloud_diffs_east.pdf}
    \caption{a nice plot}
\end{figure}
%%%%%
% 6 %
%%%%%
\chapter{Conclusion and final comments}
Let's start citing \citep[p.~22]{canovan:2002}

\bibliography{refs}

\end{document}